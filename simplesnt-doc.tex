\documentclass{simplesnt}

%% Helpful packages
\usepackage{simplebnf}[2022/05/08]
\RenewDocumentCommand{\SimpleBNFDefEq}{}{\ensuremath{⩴}}

\usepackage{tabularray}
\UseTblrLibrary{booktabs}

\usepackage{metalogo} % For XeTeX logo
\usepackage{kotex-logo}

\newcommand*{\presentfont}[1]{{\fontspec{#1}#1}}

\title{\texttt{simpleSnT}}
\subtitle{A simple package to create show and tell slides}
\author{Jay Lee}

\begin{document}
\maketitle

\begin{frame}
  \frametitle{목차}
  \tableofcontents
\end{frame}

\section{Prerequisites}

\begin{frame}[c]
  \frametitle{\TeX\ Engine}

  You are expected to use \XeLaTeX\ to use this class.\footnote{Tectonic is supported as well.}
  Use of other engines is not officially supported.

  \koTeX\ encourages the use of modern \TeX\ engines, and \XeLaTeX\ is considered the de-facto standard for Korean \TeX\ users.\footnote{http://wiki.ktug.org/wiki/wiki.php/ModernLaTeX}
\end{frame}


\begin{frame}[c]
  \frametitle{Fonts}
  Basically, the following fonts are required:
  {\footnotesize
  \begin{itemize}
    \item \presentfont{Noto Serif CJK KR},
    \item {\fontspec{KoPubWorldDotum_Pro}KoPubWorldDotum\_Pro},
    \item \presentfont{D2Coding},
    \item \presentfont{Libertinus Serif},
    \item \presentfont{Libertinus Sans},
    \item \presentfont{JuliaMono},
    \item \presentfont{Libertinus Math},
    \item \presentfont{TeX Gyre Pagella Math}, and
    \item \presentfont{Asana Math}.
  \end{itemize}
  }
  You probably have most of them installed already with your local \TeX\ distribution.
  However, if you pass \texttt{no-fonts} option to the class, i.e., \texttt{\textbackslash documentclass[no-fonts]\{simplesnt\}}, the class will not load any fonts.
\end{frame}


\begin{frame}[c, fragile]
  \frametitle{Usage}

  Use \XeLaTeX:
\begin{verbatim}
# Run at least three times to get all the cross references correct
xelatex doc.tex
xelatex doc.tex
xelatex doc.tex
\end{verbatim}

  The above is just an example; You are free to use other build tools based on \texttt{latexmk}, GNU Make, Tectonic, etc.
\end{frame}


\section{(Opinionated) Examples}

\begin{frame}[c, fragile]
  \frametitle{Backus-Naur Forms}

  Use \texttt{simplebnf}, e.g.,
\begin{verbatim}
\usepackage{simplebnf}[2022/05/08]
\RenewDocumentCommand{\SimpleBNFDefEq}{}{\ensuremath{⩴}}
\end{verbatim}

  \begin{bnfgrammar}[rrclr]
    $e$ : Expression ::=
      $n$ : number
    | $x$ : variable
    | $e$; $e$ : sequence
    | $e \oplus e$ : binary op
    | $e\ e$ : application
  \end{bnfgrammar}
\end{frame}

\begin{frame}[c, fragile]
  \frametitle{Tables}

  Use \texttt{tabularray}, e.g.,
\begin{verbatim}
\usepackage{tabularray}
\UseTblrLibrary{booktabs}
\end{verbatim}

  \begin{center}
    \begin{tblr}{ccc}
      \toprule
      A & B & C \\
      \midrule
      1 & 2 & 3 \\
      4 & 5 & 6 \\
      \bottomrule
    \end{tblr}
  \end{center}
\end{frame}
\end{document}
